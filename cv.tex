%%%%%%%%%%%%%%%%%%%%%%%%%%%%%%%%%%%%%%%%%
% Medium Length Professional CV
% LaTeX Template
% Version 2.0 (8/5/13)
%
% This template has been downloaded from:
% http://www.LaTeXTemplates.com
%
% Original author:
% Trey Hunner (http://www.treyhunner.com/)
%
% Important note:
% This template requires the resume.cls file to be in the same directory as the
% .tex file. The resume.cls file provides the resume style used for structuring the
% document.
%
%%%%%%%%%%%%%%%%%%%%%%%%%%%%%%%%%%%%%%%%%

%----------------------------------------------------------------------------------------
%	PACKAGES AND OTHER DOCUMENT CONFIGURATIONS
%----------------------------------------------------------------------------------------

\documentclass{resume} % Use the custom resume.cls style
\newcommand{\ts}{\textsuperscript}
\usepackage[left=0.75in,top=0.6in,right=0.75in,bottom=0.6in]{geometry} % Document margins
\usepackage{bibentry}
\usepackage{enumitem}

\name{Han-Yun(Henry) Yeh} % Your name
%\address{123 Broadway \\ City, State 12345} % Your address
%\address{123 Pleasant Lane \\ City, State 12345} % Your secondary addess (optional)
%\address{(000)~$\cdot$~111~$\cdot$~1111 \\ john@smith.com} % Your phone number and email
\address{henryyeh034@gmail.com, +886-922463510}
\address{github.com/henry034/}
\begin{document}


%----------------------------------------------------------------------------------------
%	EDUCATION SECTION
%----------------------------------------------------------------------------------------

\begin{rSection}{Education}

{\bf National Chiao Tung University} \hfill{Hsinchu, Taiwan} \hfill{\em Sep. 2016 - June 2019}
\begin{itemize} \item {M.S. in Communications Engineering (GPA: 4.02/4.3)} \end{itemize}
{\bf National Taipei University}\hspace*{2em} \hfill{Taipei, Taiwan} \hfill{\em Sep. 2012 - June 2016}
\begin{itemize} \item {B.S. in Communications Engineering (Rank: 1\ts{st}, GPA: 3.72/4.0 )} \end{itemize}
\end{rSection}

%----------------------------------------------------------------------------------------
%	Hornors and awards SECTION
%----------------------------------------------------------------------------------------

\begin{rSection}{Hornors and Awards}
    Research
    \begin{rSubsection}{}{}{}{}
    \item {\bf 6\ts{th} Place} (2017) - IJCNLP Shared Task 2~[3] - Taipei, Taiwan
    \item {\bf 1\ts{st} Place} (2015) - NTPU CE Senior Project Competition - Taipei, Taiwan
    \item {\bf Best Paper Award} (2014) - Oriental COCOSDA~[4] - Phuket, Thailand
    \end{rSubsection}
    Academic
    \begin{rSubsection}{}{}{}{}
    \item {\bf Phi Tau Phi Award} (2016) - The Phi Tau Phi Scholastic Honor Society, Taipei Taiwan \\
    Award for top-ranked student in CE department among all classes
    \item {\bf Dean's List} (Fall '12, Fall '15) - NTPU CE Dept. - Taipei, Taiwan
    \item {\bf Scholarship} (2014) - Elytone Electronic CO., LTD. - Taipei, Taiwan
    \item {\bf Honorable Mention} (2014) - Taiwan National Collegiate Programming Contest - Taipei, Taiwan
    \end{rSubsection}
    \end{rSection}

%----------------------------------------------------------------------------------------
%	RESEARCH EXPERIENCE SECTION
%----------------------------------------------------------------------------------------
\begin{rSection}{Research Experience}
    \begin{rSubsection}{National Chiao Tung University}{Hsinchu, Taiwan}{Graduate Student/Research Assistant, Speech Processing Lab}{Sep. 2016 - Jan. 2019}
        \item Chinese pinyin to character language model using deep learning ~[1]
        \begin{itemize}[label=$-$]
            \setlength \itemsep{-0.5em}
            \item Experimented with sequence labeling (TDNN and BLSTM joint learning with word boundary prediction) and seq2seq (Transformer) models to minimize Chinese pinyin to character recognition issues
            \item Preprocessed data from Wikipedia, LDC Chinese Gigaword, and Sinica corpus by utilizing high precision CRF-based Chinese parser and rule-based G2P (character to pinyin) systems, resulting in a reduction of the character’s error rate to 5.6\%
        \end{itemize}\vspace {0.5em}
        \item Dimensional sentiment analysis for Chinese phrases (DSAP) ~[3] 
        \begin{itemize}[label=$-$]
            \setlength \itemsep{-0.5em}
            \item Achieved a mean rank of 6.5 among 24 submissions on Chinese phrases’ valence and arousal prediction problems using the proposed order-aware word2vec and BLSTM models with the CAVT (Chinese Valence-Arousal Text) corpus
        \end{itemize}\vspace {0.5em}
        \item Child Speech Impairment Supporting System
        \begin{itemize}[label=$-$]
            \setlength \itemsep{-0.5em}
            \item Collected and analyzed approximately 200 samples from children and implemented a Java GUI based corpus recording system for children with speech impediments in coordination with NTU Hospital Hsinchu
        \end{itemize}\vspace {0.5em}
    \end{rSubsection}
    \begin{rSubsection}{National Taipei University}{Taipei, Taiwan}{Research Assistant, Speech and Multimedia Signal Processing Lab}{Sep. 2012 - June 2016}
        \item {\em "An Automatic Grade Input System via Voice"}
        \begin{itemize}[label=$-$]
            \setlength \itemsep{-0.5em}
            \item Constructed a speech recognition system that featured energy-based voice activity detection and a beam-forming noise cancellation module to enter student’s grades automatically. The project was awarded 1st place in NTPU CE Senior Project Competition
        \end{itemize}\vspace{0.5em}
        \item Mandarin prosody generation~[2][4]
        \begin{itemize}[label=$-$]
            \setlength \itemsep{-0.5em}
            \item Investigated improving CRF-based base-phrase chunk features and punctuation confidence in Mandarin text-to-speech system
            \item Labeled base-phrase chunk features by using CRF-based base-phrase chunker
            \item Generated CRF-based punctuation confidence for each lexical word boundary from input text tagged with Chinese word boundaries, 
                  part of speech (POS), and base-phrase chunk to measure the likelihood of inserting a punctuation mark (PM)
            \item Applied the above features in a MLP-based prosody generator and confirmed that the RMSE for predicting logF0, syllable duration, energy level, and pause duration were reduced
        \end{itemize}\vspace{0.5em}
    \end{rSubsection}
\end{rSection}

%----------------------------------------------------------------------------------------
%	Publications
%----------------------------------------------------------------------------------------
\begin{rSection}{Publications}
    \nobibliography{bib/mybib}
    \bibliographystyle{unsrt}
    
    \begin{enumerate}[label={[\arabic*]}]
    \item \bibentry{yeh2019}
    \item \bibentry{chiang2019punctuation}
    \item \bibentry{lee2017nctu}
    \item \bibentry{hung2014investigation}
    \end{enumerate}    
\end{rSection}

%----------------------------------------------------------------------------------------
%	WORK EXPERIENCE SECTION
%----------------------------------------------------------------------------------------

\begin{rSection}{Work Experience}
    \begin{rSubsection}{IBM, Inc}{Taipei, Taiwan}{Application Developer}{June 2018 - Oct. 2018}
        \item Organized health knowledge collected from the internet using text processing techniques 
              and designed rule-based health information suggestions using Python according to user’s information with data from a wearable device or entered manually by the user
        \item Designed a backend infrastructure to collect and analyze website user behavior by means of JavaScript, Python, PHP, MongoDB, MySQL and deployed service on AWS
    \end{rSubsection}
    \begin{rSubsection2}{Application Developer Intern}{Jul. 2017 - Aug. 2017}
        \item Implemented an automated optical inspection (AOI) algorithm to 
              detect defects in circuit board labels using Python with OpenCV
    \end{rSubsection2}
\end{rSection}

%----------------------------------------------------------------------------------------
%	Teaching and Advising Experience 
%----------------------------------------------------------------------------------------
\begin{rSection}{Teaching and Advising Experience}
    \begin{rSubsection}{Teaching Assistant, \it{National Chiao Tung University}}{Sep. 2016 - Jan. 2018}{}{Hsinchu, Taiwan}
        \item "Principle of Microcomputer" course taught by Prof. Yi-Ru, Wang \hfill Sep. 2017 - Jan. 2018
        \item "Logic Design and Lab" course taught by Prof. Yi-Ru, Wang \hfill Sep. 2016 - Jan. 2017
    \end{rSubsection}
    \begin{rSubsection}{Teaching Assistant, \it{National Taipei University}}{Feb. 2015 - June 2016}{}{Taipei, Taiwan}
        \item "Physics Lab" course taught by Prof. Cheng-Yu, Chiang \hfill Feb. 2015 - June 2016
    \end{rSubsection}
\end{rSection}

%----------------------------------------------------------------------------------------
%	Activities
%----------------------------------------------------------------------------------------
\begin{rSection}{Activities}
    \begin{rSubsection}{2020 Formosa Grand Challenge}{Taipei, Taiwan}{Participant}{Nov. 2019 - Apr. 2020}
        \item Build a model and labeled data to solve various Chinese reading comprehension tasks using PyTorh and 
              developed a rule-based dialog system to solve a conversation problem with 20 different domains
    \end{rSubsection}
    \begin{rSubsection}{The 5\ts{th} Pixnet Hackathon}{Taipei, Taiwan}{Participant}{Aug. 2018}
        \item Developed a web-based application to find popular dishes by using Google AIY Voice Kit and information from text mining 200,000 Pixnet food blogs
    \end{rSubsection}
    \begin{rSubsection}{NTPU CE Student Association}{Taipei, Taiwan}{Director of Information Management}{Sep. 2013 - June 2015}
        \item Managed student information system and designed advertising photo to promote the activities of the student association
        \item Organized procedures, activities, and lectures for freshman orientation
    \end{rSubsection}
    \begin{rSubsection2}{Leader of Curriculum-design Section}{Feb. 2015}
        \item Designed curriculum to build a DIY IR remote, speaker/microphone, and PC customization for the first NTPU CE winter camp
    \end{rSubsection2}
    \begin{rSubsection}{The International Collegiate Programming Contest}{Jakarta, Indonesia}{Participant}{Dec. 2014}
        \item Placed 29\ts{th} out of 70 teams at the ACM-ICPC Asia Jakarta Regional Contest
    \end{rSubsection}
\end{rSection}

%----------------------------------------------------------------------------------------
%	TECHNICAL STRENGTHS SECTION
%----------------------------------------------------------------------------------------

\begin{rSection}{Technical Strengths}

\begin{tabular}{ @{} >{\bfseries}l @{\hspace{6ex}} l }
Computer Languages & Python, C, Java, Javascript, PHP \\
Databases & MySQL, PostgreSQL \\
Frameworks & TensorFlow, Pytorch \\
Package & OpenCV \\
Cloud service & AWS
\end{tabular}

\end{rSection}

%----------------------------------------------------------------------------------------
%	EXAMPLE SECTION
%----------------------------------------------------------------------------------------

%\begin{rSection}{Section Name}

%Section content\ldots

%\end{rSection}

%----------------------------------------------------------------------------------------

\end{document}
