%%%%%%%%%%%%%%%%%%%%%%%%%%%%%%%%%%%%%%%%%
% Medium Length Professional CV
% LaTeX Template
% Version 2.0 (8/5/13)
%
% This template has been downloaded from:
% http://www.LaTeXTemplates.com
%
% Original author:
% Trey Hunner (http://www.treyhunner.com/)
%
% Important note:
% This template requires the resume.cls file to be in the same directory as the
% .tex file. The resume.cls file provides the resume style used for structuring the
% document.
%
%%%%%%%%%%%%%%%%%%%%%%%%%%%%%%%%%%%%%%%%%

%----------------------------------------------------------------------------------------
%	PACKAGES AND OTHER DOCUMENT CONFIGURATIONS
%----------------------------------------------------------------------------------------

\documentclass{resume} % Use the custom resume.cls style
\newcommand{\ts}{\textsuperscript}
\usepackage[left=0.75in,top=0.6in,right=0.75in,bottom=0.6in]{geometry} % Document margins
\usepackage{bibentry}
\usepackage{enumitem}

\name{Han-Yun(Henry) Yeh} % Your name
%\address{123 Broadway \\ City, State 12345} % Your address
%\address{123 Pleasant Lane \\ City, State 12345} % Your secondary addess (optional)
%\address{(000)~$\cdot$~111~$\cdot$~1111 \\ john@smith.com} % Your phone number and email
\address{henryyeh034@gmail.com, +886-922463510}
\address{github.com/henry034/}
\begin{document}


%----------------------------------------------------------------------------------------
%	EDUCATION SECTION
%----------------------------------------------------------------------------------------

\begin{rSection}{Education}

{\bf National Chiao Tung University} \hfill{Hsinchu, Taiwan} \hfill{\em Sep. 2016 - June 2019}
\begin{itemize} \item {M.S. in Communications Engineering (GPA: 4.02/4.3)} \end{itemize}
{\bf National Taipei University}\hspace*{2em} \hfill{Taipei, Taiwan} \hfill{\em Sep. 2012 - June 2016}
\begin{itemize} \item {B.S. in Communications Engineering (Rank: 1\ts{st}, GPA: 3.72/4.0 )} \end{itemize}
\end{rSection}

%----------------------------------------------------------------------------------------
%	Hornors and awards SECTION
%----------------------------------------------------------------------------------------

\begin{rSection}{Hornors and Awards}
    Research
    \begin{rSubsection}{}{}{}{}
    \item {\bf 6\ts{th} Place} (2017) - IJCNLP Shared Task 2~[3] - Taipei, Taiwan
    \item {\bf 1\ts{st} Place} (2015) - NTPU CE Senior Project Competition - Taipei, Taiwan
    \item {\bf Best Paper Award} (2014) - Oriental COCOSDA~[4] - Phuket, Thailand
    \end{rSubsection}
    Academic
    \begin{rSubsection}{}{}{}{}
    \item {\bf Phi Tau Phi Award} (2016) - The Phi Tau Phi Scholastic Honor Society, Taipei Taiwan \\
    Award for top-ranked student in CE department among all classes
    \item {\bf Dean's List} (Fall '12, Fall '15) - NTPU CE Dept. - Taipe, Taiwan
    \item {\bf Scholarship} (2014) - Elytone Electronic CO., LTD. - Taipei, Taiwan
    \item {\bf Honerable Mention} (2014) - Taiwan National Collegiate Programming Contest - Taipei, Taiwan
    \end{rSubsection}
    \end{rSection}

%----------------------------------------------------------------------------------------
%	RESEARCH EXPERIENCE SECTION
%----------------------------------------------------------------------------------------
\begin{rSection}{Research Experience}
    \begin{rSubsection}{National Chiao Tung University}{Hsinchu, Taiwan}{Graduate Student/Research Assistant, Speech Processing Lab}{Sep. 2016 - Jan. 2019}
        \item Research Topic: Chinese pinyin to character language model using deep learning approach ~[1]
        \begin{itemize}[label=$-$]
            \setlength \itemsep{-0.5em}
            \item Experimented sequence labeling (TDNN+BLSTM, joint learning with word boundary prediction) and seq2seq (Transformer) model on Chinese pinyin to character problem. 
            \item Preprocessed data from text corpus including wikipedia, LDC Chinese Gigaword, Sinica corpus,\dots through high precision CRF-based Chinese parser and rule-based G2P (character to pinyin) system.
            \item Reduced character error rate to 5.6\%.
        \end{itemize}\vspace {0.5em}
        \item Dimensional sentiment analysis for Chinese phrases (DSAP)~[3] 
        \begin{itemize}[label=$-$]
            \setlength \itemsep{-0.5em}
            \item Achieved mean rank 6.5 among 24 submissions on Chinese phrases' valence and arousal prediction problem using proposed order-awared word2vec and BLSTM model on CAVT(Chinese Valence-Arousal Text) corpus.
        \end{itemize}\vspace {0.5em}
        \item Child Speech Impairment Supporting System
        \begin{itemize}[label=$-$]
            \setlength \itemsep{-0.5em}
            \item Implemented corpus recording system (for speech impairment child) GUI using Java in the cooperation project with NTU Hospital Hsinchu. Collected and analyzed about 200 children's sample.
        \end{itemize}\vspace {0.5em}
    \end{rSubsection}
    \begin{rSubsection}{National Taipei University}{Taipei, Taiwan}{Research Assistant, Speech and Multimedia Signal Processing Lab}{Sep. 2012 - June 2016}
        \item Senior project: {\em "An Automatic Grade Input System via Voice"}
        \begin{itemize}[label=$-$]
            \setlength \itemsep{-0.5em}
            \item Constructed speech recognition system including energy-based voice activity detection and beamforming noise cancellation module to enter student's name and score automatically. Obtained 1\ts{st} place in NTPU CE Senior Project Competition.
        \end{itemize}\vspace{0.5em}
        \item Mandarin prosody generation~[2][4]
        \begin{itemize}[label=$-$]
            \setlength \itemsep{-0.5em}
            \item Investigated the improvement of CRF-based base-phrase chunk feature and punctuation confidence in Mandarin text-to-speech system. 
            \item Labeled base-phrase chunk feature by CRF-based base-phrase chunker. 
            \item Generated CRF-based punctuation confidence for each lexical word boundary from input text tagged with Chinese word boundary, part of speech (POS) and base-phrase chunk, 
                  measuring the likelihood of inserting a punctuation mark (PM). 
            \item Applied above feature in MLP-based prosody generator and confirmed that the RMSE for predicting logF0, syllable duration, energy level and pause duration are all reduced.
        \end{itemize}\vspace{0.5em}
    \end{rSubsection}
\end{rSection}

%----------------------------------------------------------------------------------------
%	Publications
%----------------------------------------------------------------------------------------
\begin{rSection}{Publications}
    \nobibliography{bib/mybib}
    \bibliographystyle{unsrt}
    
    \begin{enumerate}[label={[\arabic*]}]
    \item \bibentry{yeh2019}
    \item \bibentry{chiang2019punctuation}
    \item \bibentry{lee2017nctu}
    \item \bibentry{hung2014investigation}
    \end{enumerate}    
\end{rSection}

%----------------------------------------------------------------------------------------
%	WORK EXPERIENCE SECTION
%----------------------------------------------------------------------------------------

\begin{rSection}{Work Experience}
    \begin{rSubsection}{IBM, Inc}{Taipei, Taiwan}{Application Developer}{June 2018 - Oct. 2018}
        \item Organized health knowledge collected from internet using text processing techniques
        and designed rule-based health information suggestion according to user's information which includes the data from wearable device or entered by user, using Python.
        \item Designed backend infrastructure to collect, analyzed website user behavior data with Javascript, Python, PHP, MongoDB, MySQL and deployed service on AWS.
    \end{rSubsection}
    \begin{rSubsection2}{Application Developer Intern}{Jul. 2017 - Aug. 2017}
        \item Implemented automated optical inspection (AOI) algorithm of defect detection for the label on circuit board 
        including misprint, broken label,\dots, using Python with OpenCV package.
    \end{rSubsection2}
\end{rSection}

%----------------------------------------------------------------------------------------
%	Teaching and Advising Experience 
%----------------------------------------------------------------------------------------
\begin{rSection}{Teaching and Advising Experience}
    \begin{rSubsection}{Teaching Assistant, \it{National Chiao Tung University}}{Sep. 2016 - Jan. 2018}{}{Hsinchu, Taiwan}
        \item "Principle of Microcomputer" course taught by Prof. Yi-Ru, Wang \hfill Sep. 2017 - Jan. 2018
        \item "Logic Design and Lab" course taught by Prof. Yi-Ru, Wang \hfill Sep. 2016 - Jan. 2017
    \end{rSubsection}
    \begin{rSubsection}{Teaching Assistant, \it{National Taipei University}}{Feb. 2015 - June 2016}{}{Taipei, Taiwan}
        \item "Physics Lab" course taught by Prof. Cheng-Yu, Chiang \hfill Feb. 2015 - June 2016
    \end{rSubsection}
\end{rSection}

%----------------------------------------------------------------------------------------
%	Activities
%----------------------------------------------------------------------------------------
\begin{rSection}{Activities}
    \begin{rSubsection}{2020 Formosa Grand Challenge}{Taipei, Taiwan}{Participant}{Nov. 2019 - Apr. 2020}
        \item Build model and labeled data to solve various Chinese reading comprehension task using Pytorh 
        and build rule-based dialog system to solve conversation problem with 20 different domains.
    \end{rSubsection}
    \begin{rSubsection}{2018 5\ts{th} Pixnet Hackthon}{Taipei, Taiwan}{Participant}{Aug. 2018}
        \item Developed a device and management website for answering popular dishes using Google AIY Voice Kit and information from text mining on 200,000 Pixnet food blogs.
    \end{rSubsection}
    \begin{rSubsection}{NTPU CE Student Association}{Taipei, Taiwan}{Director of Information Management}{Sep. 2013 - June 2015}
        \item Managed student information system and designed advertising photo for promoting the activities of student association.
        \item Instructor of orientation events for freshman.
    \end{rSubsection}
    \begin{rSubsection2}{Leader of Curriculum-design Section}{Feb. 2015}
        \item Designed curriculum including IR remoter DIY, speaker/microphone DIY and PC building of 1\ts{st} NTPU CE winter camp.
    \end{rSubsection2}
    \begin{rSubsection}{The International Collegiate Programming Contest}{Jakarta Indonesia}{Participant}{Dec. 2014}
        \item Placed 29\ts{th} among 70 teams in 2014 ACM-ICPC Asia Jakarta Regional Contest. Solved 4 of 11 problems.
    \end{rSubsection}
\end{rSection}

%----------------------------------------------------------------------------------------
%	TECHNICAL STRENGTHS SECTION
%----------------------------------------------------------------------------------------

\begin{rSection}{Technical Strengths}

\begin{tabular}{ @{} >{\bfseries}l @{\hspace{6ex}} l }
Computer Languages & Python, C, Java, Javascript, PHP \\
Databases & MySQL, PostgreSQL \\
Frameworks & TensorFlow, Pytorch \\
Package & OpenCV \\
Cloud service & AWS
\end{tabular}

\end{rSection}

%----------------------------------------------------------------------------------------
%	EXAMPLE SECTION
%----------------------------------------------------------------------------------------

%\begin{rSection}{Section Name}

%Section content\ldots

%\end{rSection}

%----------------------------------------------------------------------------------------

\end{document}
